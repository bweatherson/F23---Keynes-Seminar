% Options for packages loaded elsewhere
\PassOptionsToPackage{unicode}{hyperref}
\PassOptionsToPackage{hyphens}{url}
\PassOptionsToPackage{dvipsnames,svgnames,x11names}{xcolor}
%
\documentclass[
  11pt,
  letterpaper,
  DIV=11,
  numbers=noendperiod,
  oneside]{scrartcl}

\usepackage{amsmath,amssymb}
\usepackage{iftex}
\ifPDFTeX
  \usepackage[T1]{fontenc}
  \usepackage[utf8]{inputenc}
  \usepackage{textcomp} % provide euro and other symbols
\else % if luatex or xetex
  \ifXeTeX
    \usepackage{mathspec} % this also loads fontspec
  \else
    \usepackage{unicode-math} % this also loads fontspec
  \fi
  \defaultfontfeatures{Scale=MatchLowercase}
  \defaultfontfeatures[\rmfamily]{Ligatures=TeX,Scale=1}
\fi
\usepackage{lmodern}
\ifPDFTeX\else  
    % xetex/luatex font selection
  \setmainfont[Scale = MatchLowercase]{Scala Pro}
  \setsansfont[]{Scala Sans Pro}
  \ifXeTeX
    \setmathfont(Digits,Latin,Greek)[]{Scala Pro}
  \else
    \setmathfont[]{Scala Pro}
  \fi
\fi
% Use upquote if available, for straight quotes in verbatim environments
\IfFileExists{upquote.sty}{\usepackage{upquote}}{}
\IfFileExists{microtype.sty}{% use microtype if available
  \usepackage[]{microtype}
  \UseMicrotypeSet[protrusion]{basicmath} % disable protrusion for tt fonts
}{}
\makeatletter
\@ifundefined{KOMAClassName}{% if non-KOMA class
  \IfFileExists{parskip.sty}{%
    \usepackage{parskip}
  }{% else
    \setlength{\parindent}{0pt}
    \setlength{\parskip}{6pt plus 2pt minus 1pt}}
}{% if KOMA class
  \KOMAoptions{parskip=half}}
\makeatother
\usepackage{xcolor}
\usepackage[left=1in,marginparwidth=2.0666666666667in,textwidth=4.1333333333333in,marginparsep=0.3in]{geometry}
\setlength{\emergencystretch}{3em} % prevent overfull lines
\setcounter{secnumdepth}{-\maxdimen} % remove section numbering
% Make \paragraph and \subparagraph free-standing
\ifx\paragraph\undefined\else
  \let\oldparagraph\paragraph
  \renewcommand{\paragraph}[1]{\oldparagraph{#1}\mbox{}}
\fi
\ifx\subparagraph\undefined\else
  \let\oldsubparagraph\subparagraph
  \renewcommand{\subparagraph}[1]{\oldsubparagraph{#1}\mbox{}}
\fi


\providecommand{\tightlist}{%
  \setlength{\itemsep}{0pt}\setlength{\parskip}{0pt}}\usepackage{longtable,booktabs,array}
\usepackage{calc} % for calculating minipage widths
% Correct order of tables after \paragraph or \subparagraph
\usepackage{etoolbox}
\makeatletter
\patchcmd\longtable{\par}{\if@noskipsec\mbox{}\fi\par}{}{}
\makeatother
% Allow footnotes in longtable head/foot
\IfFileExists{footnotehyper.sty}{\usepackage{footnotehyper}}{\usepackage{footnote}}
\makesavenoteenv{longtable}
\usepackage{graphicx}
\makeatletter
\def\maxwidth{\ifdim\Gin@nat@width>\linewidth\linewidth\else\Gin@nat@width\fi}
\def\maxheight{\ifdim\Gin@nat@height>\textheight\textheight\else\Gin@nat@height\fi}
\makeatother
% Scale images if necessary, so that they will not overflow the page
% margins by default, and it is still possible to overwrite the defaults
% using explicit options in \includegraphics[width, height, ...]{}
\setkeys{Gin}{width=\maxwidth,height=\maxheight,keepaspectratio}
% Set default figure placement to htbp
\makeatletter
\def\fps@figure{htbp}
\makeatother

\setlength\heavyrulewidth{0ex}
\setlength\lightrulewidth{0ex}
\KOMAoption{captions}{tableheading}
\makeatletter
\@ifpackageloaded{caption}{}{\usepackage{caption}}
\AtBeginDocument{%
\ifdefined\contentsname
  \renewcommand*\contentsname{Table of contents}
\else
  \newcommand\contentsname{Table of contents}
\fi
\ifdefined\listfigurename
  \renewcommand*\listfigurename{List of Figures}
\else
  \newcommand\listfigurename{List of Figures}
\fi
\ifdefined\listtablename
  \renewcommand*\listtablename{List of Tables}
\else
  \newcommand\listtablename{List of Tables}
\fi
\ifdefined\figurename
  \renewcommand*\figurename{Figure}
\else
  \newcommand\figurename{Figure}
\fi
\ifdefined\tablename
  \renewcommand*\tablename{Table}
\else
  \newcommand\tablename{Table}
\fi
}
\@ifpackageloaded{float}{}{\usepackage{float}}
\floatstyle{ruled}
\@ifundefined{c@chapter}{\newfloat{codelisting}{h}{lop}}{\newfloat{codelisting}{h}{lop}[chapter]}
\floatname{codelisting}{Listing}
\newcommand*\listoflistings{\listof{codelisting}{List of Listings}}
\makeatother
\makeatletter
\makeatother
\makeatletter
\@ifpackageloaded{caption}{}{\usepackage{caption}}
\@ifpackageloaded{subcaption}{}{\usepackage{subcaption}}
\makeatother
\makeatletter
\@ifpackageloaded{sidenotes}{}{\usepackage{sidenotes}}
\@ifpackageloaded{marginnote}{}{\usepackage{marginnote}}
\makeatother
\ifLuaTeX
  \usepackage{selnolig}  % disable illegal ligatures
\fi
\IfFileExists{bookmark.sty}{\usepackage{bookmark}}{\usepackage{hyperref}}
\IfFileExists{xurl.sty}{\usepackage{xurl}}{} % add URL line breaks if available
\urlstyle{same} % disable monospaced font for URLs
\hypersetup{
  pdftitle={Week 9: The General Theory},
  pdfauthor={Brian Weatherson},
  colorlinks=true,
  linkcolor={black},
  filecolor={Maroon},
  citecolor={Blue},
  urlcolor={Blue},
  pdfcreator={LaTeX via pandoc}}

\title{Week 9: The General Theory}
\author{Brian Weatherson}
\date{2023-11-07}

\begin{document}
\maketitle
\begin{abstract}
Starting on Keynes's greatest book, the 1936 General Theory of
Employment, Interest and Money. Naturally, we're starting at both ends,
and working towards the middle.
\end{abstract}
\section{Nothing More Powerful than
Ideas}\label{nothing-more-powerful-than-ideas}

Let's start at the end, with the remarkable ending of the book. It's
possibly the most quoted part of the book, joining the \emph{Tractatus}
and Hume's \emph{Enquiry} perhaps in books whose endings are among the
most memorable things about them.

Keynes gives a rather grandiose claim about the role of ``economic and
political philosophers'' in history. They, we?, turn out to be the most
important people in the world. Even if our ideas only end up getting
implemented by thugs with guns, just which direction the thuggery takes
is, according to Keynes, determined by the philosophers.\sidenote{\footnotesize Eric
  Schlisser pointed out to me that a similar idea is presented in A. V.
  Dicey's 1905 book \emph{Lectures on the Relation Between Law and
  Public Opinion in England During the Nineteenth Century}. If Keynes is
  right here, someone should have said something similar when he was
  about 22, so this tracks.} It's a big role!

Is he correct about this? The first thing to think about might be the
madmen in authority at the time Keynes was writing. It wasn't a great
time for European political leadership: Stalin in Russia, Hitler in
Germany\sidenote{\footnotesize We could perhaps talk about the preface to the German
  edition of the \emph{General Theory}, which is notorious for not being
  quite as anti-Nazi as one might hope.}, Mussolini in Italy, Franco in
Spain. Are they just following recent academic scribblers?

Franco doesn't look like a great case for Keynes. He's just following a
fairly well established playbook of balancing the power of the powerful
groups he cared about: the army, the church, and big business. If
there's a grand theory there beyond look after one's own, I'm not sure
what it is.

Hitler and Mussolini are a bit more interesting. How much is Keynes
thinking that Nietzsche and/or Pareto (or similar figures) are
responsible for what they are doing? Or is he even thinking of Heidegger
here; that's unlikely given the timeline. But there's plausibly a case
that the distinct way that Hitler and/or Mussolini ruled is a function
of various theories.\sidenote{\footnotesize Mussolini has the added complication that,
  like every military leader from Italy, everything harkens back to the
  Roman Empire. But, Keynes might say, it's always a version of Rome;
  what matters isn't what Caeser or Trajan did, but what Mussolini and
  his friends were taught they did.}

But Stalin might be the most interesting case. On the one hand, it's
hard not to read this passage as a swipe at the Marxist theory of
history. The idea that `vested interests' determine history is a (crude,
uncharitable) summary of what Marx thought. And Stalin was obviously the
only Marxist among the four I mentioned.

Now on the one hand, Stalin was in a sense the least doctrinaire of the
Bolsheviks, indeed almost of any of the left-wing leaders in 1917. It's
not clear how much he even \emph{read} Marx. He certainly wasn't going
to be like the Mensheviks and put off the revolution because history
said it couldn't happen yet. He was even less likely to bother with
writing his own treatise on how Marx had to be updated a la Trotsky. So
does that tell against Keynes?

I don't think it does. Stalin was a very doctrinaire leader, even more
so than Lenin ever was. Collectivisation wasn't Stalin's idea, but it
very much became his policy. I don't think Lenin did much
collectivisation at all. And Stalin kept at it, well after it had become
clear that it was a catastrophic failure. I don't even think it really
satisfied the material interests of the group Stalin allegedly
represented. But he'd been a Bolshevik for as long as he could remember,
and this was Bolshevik policy, and he was determined to put it into
practice.

It's also worth noting the notable failures of Marxist theories in
predicting what would happen after World War I. These theories make some
fairly clear predictions. First, revolution should come first to the
most developed countries. Second, there needs to be a bourgeoisie
revolution (a la 1789) before there is a proletarian revolution (a la
1792 or 1871). None of that happened. The German revolution failed
almost at inception. The bourgeoisie revolution in Russia was basically
a failure, but the proletarian revolution happened anyway.

That said, if there is a problem for Keynes's theory of history it might
come from the leader he most admired at this time: FDR. It's obviously
not true that the New Deal came from economic and political philosophers
that FDR read growing up. No one was advocating that then! Keynes
himself wasn't advocating anything like it.

Indeed, there's some tension between the picture we get in chapter 23
and the one we get at the end of 24. In chapter 23 one of the running
themes is that the politicians kind of know what they are doing. Not in
any deep sense of course; their theories about why things work are
typically off in every possible respect. But they do, as a group, have
an intuitive sense of what needed to be done. And the plans are at least
directionally correct.

The gold standard era politicians who cared a lot about the balance of
trade were, thinks Keynes, basically doing the right thing given the
constraints they had to work with.\sidenote{\footnotesize That's not because there's
  anything special about the balance of trade. But because of the
  details of how the gold standard worked, a better balance of trade
  meant better, i.e., lower, interest rates. And that's what really
  mattered.} Those politicians weren't just following the ravings of an
academic scribbler some years back. All the academic scribblers were
united in their opposition to such a policy.

So while I'm somewhat sympathetic to Keynes's views about the madmen in
authority, and very sympathetic to taking shots at the Marxist theory of
history, for better or worse I don't think it's our ideas that
ultimately rule the world. Sometimes it's the animal cunning of the
politicians.

\section{The Classical Theory}\label{the-classical-theory}

I suspect everyone here knows this as well as I do, but I wanted to set
out what I take to be the `classical' theory of unemployment; the one
that Keynes is responding to in chapter 2. (And, frankly, the rest of
the book.)

Start with the theory about the market for some particular good. Let's
use milk as our example. Here's the simple textbook version.

Start by holding fixed the income of everyone in the community, and the
price of all other goods in the economy. Given those facts, there is an
amount of milk each person would buy for each possible price of milk.
Summing these up, we get the amount of milk that the community will buy
for each possible price of milk. That is, there is a function D from
prices to quantities such that D(p) = q means that at price p, we'll
collectively buy quantity q of milk. And D is downward-sloping; at
higher prices we buy other things instead.

There is a similar function S(p) = q from prices of milk to the amount
of milk supplied. And this is upward sloping. As the price of milk
rises, more people get into the dairy business, and more milk is
supplied. But as the price falls, farms go bankrupt, and the supply
shrinks.

So that means that there is a unique point where D(p) = S(p). And that
price is going to be the price of milk, and the quantity of milk
bought/sold is D(p) = S(p).

The classical theory says that labour is like this as well. In this case
the price is wages, so let's write it as w. Again, there is a D function
and an S function, though one thing to note already is that who
determines D and S is flipped from the usual case. D is a function of
what business does; S is a function of what normal people do. That's the
other way around to normal, and I guess this matters.

Anyway, D(w) is the amount of labour demanded (by businesses) at wage w.
The higher w is, the less labour they'll demand. If wages are high, you
stop opening your restaurant at low traffic times, for
example.\sidenote{\footnotesize That's happened a bit in Ann Arbor recently.} If wages
are low, you'll employ people for things that might not be profitable in
normal times.\sidenote{\footnotesize Think of all those Great Recession era delivery
  services that started up.} So D slopes down, just like the normal
case. And, says the classical theory, S slopes up. Once w gets low
enough, some people will decide that it's not worth getting out of bed
for less than \$10,000, and stay in bed. So we've got the same
conditions that are met in the milk case. We have two functions with
opposite slopes, so they will have a unique intersection where D(w) =
S(w). That value of w will be the wage level, and D(w) will be the
amount of labour demanded. If it's less than the size of the employable
population, there will be unemployment. But note that this could all be
fixed with a shift in S; if people weren't so reluctant to work for
lower wages, we could have a higher employment equilibrium.

\section{Problems with the Classical
Theory}\label{problems-with-the-classical-theory}

Start with why this is a less than optimal theory of the price of milk.
The theory is what's called a \textbf{partial} equilibrium theory. It
doesn't try and balance all the forces in the economy; just the forces
that directly affect the price of milk. But this leaves out all sorts of
complications that could, in principle, be important.\sidenote{\footnotesize General
  equilibrium theory tries to solve for all of these forces at once, and
  given some truly heroic assumptions you can prove the existence of a
  unique general solution. But the assumptions are really rather absurd,
  and unlike partial equilibrium theory, there is no particular reason
  to think that having the assumptions be nearly satisfied will make the
  result nearly correct.} In particular, and for our interest, it leaves
out complications caused by the existence of complementary goods, and
possible macro effects of the particular market.

Complementary goods are a ubiquitous feature of our lives. They are
goods that have more value if other goods are available. Gasoline is a
complement to cars, for example. Or, to bring it back to milk, milk and
corn flakes are complementary. If the price of corn flakes falls, demand
for milk will go up. So if the solution to the equation for milk had a
material effect on the price of corn flakes (e.g., because milk was also
used as an input to corn flakes), the partial equilibrium would really
not be a general equilibrium.

The weird things that happen with complementary goods are incredibly
important parts of the current economy. I write these notes using two
main bits of software, VS Code and RStudio. Both of these are provided,
for free, by for profit companies. And they are provided because they
are complements to other goods those companies provide. In RStudio's
case, at least as I understand the business model, the complementary
good is paid customer support for RStudio. It's not exactly trivial to
do a partial equilibrium analysis of a product like that!

The other issue is macro-economic consequences. I guess these days no
one thinks this is a particularly realistic case, but when I was a
student we were taught to worry about so-called Giffin goods. Consider
the following just-so story about the market for some basic food in a
society that is just above the subsistence line. If the price of that
basic food goes up, demand for it might go \textbf{up}. The reason is
that people who were just above subsistence level, and hence able to
afford alternatives to the basic food, will be thrust back into a state
where they can afford nothing but the basic food. This does seem
possible in theory, though I gather no one thinks it has ever occurred
in practice. But what matters to us is the general possibility; the
resolution of forces within some market might have wide-ranging effects.

Both of these things matter, thinks Keynes, in the labour market.

Labour is a complement to capital. In Keynes's model, simply employing
people is actually not something commonly done. Rather, people
\emph{start businesses}, and a business requires labour and capital.
Ultimately, Keynes's explanation of the market for labour, which is the
thing in the title of the book, is that it is kind of a side-effect of
the market for capital, i.e., investment goods. Now we might worry
somewhat about this. Sometimes unemployment goes down because of new
investments; e.g., a factory gets built. But sometimes unemployment goes
up because of new investments. That happens at a trivial level with
things like ATMs, but it happened at a much larger scale in agriculture.
I'm worried these things affect Keynes's story.

More importantly, labour has macro-economic effects. Labour is such a
big part of the economy that changes in wages mean changes in the
overall price level, and in the overall demand for goods.\sidenote{\footnotesize Strictly
  speaking this is true for every good, but it's pretty much ignorable
  in the case of corn flakes, and even milk.} In chapter 2, we get one
direction of this. If the wage goes down, prices will fall, but that
might mean the \emph{real} wage hasn't changed much at all. Keynes is a
bit slippery here, and doesn't work out whether the magnitudes really
work. And ultimately I'm not sure that what he says in chapter 2 is
really what he thinks. The big problem with a cut in wages is that it
will lead to less demand, which leads to less investment, which means
fewer things for labour to complement, which means possibly a
\emph{fall} in the demand for labour. But that's getting ahead of
ourselves. For now the worry is that a change in wages will lead to a
change in demand for good produced by workers (what Keynes calls
wage-goods), and that will change the demand curve for labour. So there
is no guarantee that there can be an equilibrium.

\section{Keynes's Positive Theory}\label{keyness-positive-theory}

This gets more complicated, and the exegesis is more controversial. But
here I think is the view in the very broadest outline.

Labour is a compliment to capital. Ultimately, the demand for labour is
a by-product of the demand for investment. And the demand for investment
is determined by two things:

\begin{enumerate}
\def\labelenumi{\arabic{enumi}.}
\tightlist
\item
  How much investors expect speculative investments to yield.
\item
  How much investors can get from things other than investments, in
  particular from money.
\end{enumerate}

And so we get a lot of chapters in the middle of the book on these two
factors, on what determines the expectation of investment yield, and on
what determines the rate of interest. Both of these are, I think,
philosophically interesting, because they both connect to the somewhat
distinctive way in which Keynes is thinking about rational choice.

That said, Keynes doesn't reject the classical view quite as much as one
might perhaps have expected. He does think that in a slump, solving the
problem will require a fall in real wages. This isn't because he
believes in a supply-demand story about labour. Rather, it's because (I
think) lower real wages leads to higher expected investment return,
leads to higher investment, leads to more jobs.

So why don't we end back with the policy prescription of the classical
view: the solution to a slump is to cut wages? There are a few reasons,
but the biggest is that word \textbf{real}. Keynes thinks that you need
to have a real wage cut to solve the problem, and cutting money wages
won't do the trick.

But here we get into one of the more common interpretations, I think
misinterpretations, of Keynes. One way a lot of economists (mostly but
not exclusive right-leaning) read Keynes is as giving a story of what an
economy with \emph{sticky wages} looks like. Some of them agree
(especially the not so right-leaning ones) that he's given a very good
diagnosis of what a sticky wage economy looks like. But, they say, this
shows that the title of the book is completely misleading. He hasn't
analysed the general case at all, just the weird special case where
wages are sticky.

Now one thing to note immediately is that wages very much are sticky at
least in the short term. I don't think I've had my (nominal) pay change
during the academic year once in my career. You make a deal in dollars,
and that deal is paid out in dollars. It's not that special a case.

But note that the more right-leaning economists who think this see
sticky wages as a double failing.

First, it's a failing on the part of the workers who don't realise that
inflation is the same thing as a wage cut. These economists think that
at best Keynes is offering governments a one-time trick. Do a quick
inflation, no one will notice that their real wages are cut, and it will
be just like the prescription we wanted. But you can't fool people the
same way twice. If you keep doing this, workers will start demanding
things like Cost of Living Allowances, i.e., automatic inflation
adjustments. And then it will all fail.\sidenote{\footnotesize To be totally clear
  about this, the story continues by claiming this is \emph{exactly}
  what happened in the 1970s, so it explains why Keynesian economics
  seemed to work for a while, but also why it failed.}

Second, it's a failing on the part of the government. If you'd just be
tough enough with the unions, like Thatcher with the miners or Reagan
with the air-traffic controllers, you break union power, and you don't
need sticky wages any more.

I think this is a complete misreading of what Keynes is up to in chapter
two.

\section{Rationality and Sticky
Wages}\label{rationality-and-sticky-wages}

There are two arguments Keynes gives for why sticky wages are a good
thing. The second\sidenote{\footnotesize And, to be clear, the one he says is more
  important, even though I'm spending less time on it because it's less
  philosophically relevant} is that if wages were too responsive to
price moves, we'd never get to equilibrium. Every minor slump would lead
to a race to the bottom to find a point where wages were low enough;
every boom would lead to an out-of-control wage-price spiral.

The first is more interesting to us. Keynes makes the workers sound like
folks playing a Schelling-style coordination game.\sidenote{\footnotesize If not
  everyone knows what this is, I'll stop and explain, because it's a
  really useful concept.} If there's a slump, what's best for workers
collectively is that real wages are slightly cut. But for any individual
worker, having their wages cut by more than other wages are cut is
unnecessary. So there is a coordination problem. We need to find a way
to cut everyone's real wages, no more nor less than is needed to get out
of the slump, and to do so in a way that shares the pain as evenly as
possible.\sidenote{\footnotesize Actually the individual workers need not care about
  fairness; all they need to care about is that their probability of
  losing their job is low, and that their wages are as high as possible
  given the first constraint.} There are a bunch of ways to bring this
about. If we need to cut real wages by 5\%, any combination of cutting
nominal wages by x\%, and having y\% inflation, where x+y=5, will do it.
If everyone else would agree to an x\% nominal cut, it's in the interest
of these workers to agree to it as well. But if the aim is to find a
value of x that we can all agree on, the natural value, the focal point
as Schelling might have put it, is 0. The simplest equilibrium of the
game the workers, collectively, are playing is that nominal wages stay
exactly where they are: no one gets a (nominal) wage cut, but no one
gets compensated for the inflation that's happening.

What I really want to stress is that if this is a coordination game,
this is completely rational behaviour. This is not a bounded rationality
model; it's a model where the relevant kind of rationality is
game-theoretic, not decision-theoretic.

\section{Keynesian Rationality}\label{keynesian-rationality}

Simplifying only a little, Keynes thinks the model of rational choice
used in standard economic models is too impoverished in three respects.
It doesn't allow for:

\begin{enumerate}
\def\labelenumi{\arabic{enumi}.}
\tightlist
\item
  Planning
\item
  Coordination
\item
  Sensitivty to uncertainty
\end{enumerate}

The first two are what you get if you move from decision theory to game
theory. The third is more distinctively Keynesian.

In the standard economic models\sidenote{\footnotesize That is, the ones we are using
  when we tell those simple stories about milk}, our consumer has an
income constraint, and a bunch of consumption goods they can purchase.
There isn't an obvious place for planning, and especially for saving.
Now we can complicate the models only a little\sidenote{\footnotesize And, to be sure,
  you start getting these complications by 200- or 300-level courses} to
allow for shifting consumption over time.

But there is an important catch. In one sense, the way we defer
consumption to the future is by buying a very special kind of good:
money. Money is a special kind of good because it's supply does not look
anything like the supply of milk, or corn flakes, or really anything
else. And one thread in the book is going to be the important features
of this market.

Planning is going to matter in another context as well. Let's say the
price of corn flakes falls. Do you buy more corn flakes? Well, maybe
not. If the price fall is evidence that the price is going to fall again
soon, then maybe the price fall will \emph{decrease} demand, because
people (who are not stupid) will see that an even better deal is just
around the corner. That's maybe not such a big deal for corn flakes, but
it is going to be very important in explaining the market for investment
goods, which as I've said a few times, is the central market.

Coordination is what we've just seen. In the simple supply-demand
models, there are infinitely many agents, and coordinating with any
finite subset of them makes no economic difference. That's, er, not the
world we actually live in. And if my reading of chapter 2 is right,
Keynes takes these coordination effects really seriously.

Finally, though it doesn't come up yet, Keynes is going to want a theory
of rationality that is sensitive to the risk/uncertainty distinction.
Exactly how the risk/uncertainty distinction matters is a question
that's very hard to get clear on, and gets into the murkiest exegetical
waters. But it's something we'll talk a lot about the next two weeks.

\section{Why Markets?}\label{why-markets}

In chapter 24 we get, by my count, three different arguments for wanting
an economy (and even society) where markets play a key role.

First, there is a very standard liberal style argument about the
relationship between markets and freedom/autonomy. Some of these
paragraphs look like they could have been written by Mill, or Friedman,
or Hayek.

Second, there is the closely related argument that totalitarian systems
are really drab. Everything looks the same as everything else, and there
is no place for `fancy'. As arguments go, we should reject Bolshevism
because the buildings are ugly and the music is boring isn't the most
high-minded, but I think he's got a point.\sidenote{\footnotesize I'm using
  `Bolshevism' as opposed to `Marxism', or `communism', let alone
  `socialism', because the opponent here really is the very specific
  system being implemented in the Soviet Union. And if I use any other
  term I'd have to get into interpretative disputes about whether
  northern European social democratic movements, which Keynes is very
  friendly to, are excluded from those terms.}

Third, there is the argument that people who are really keen on playing
with markets, who are really obsessed with money are, to put it bluntly,
kind of psychopaths. If you don't have markets, they'll start playing
with something else. And that will be worse.\sidenote{\footnotesize The premise that
  the something else will be worse is, I guess, somewhat questionable,
  given the downside effects of pollution, money laundering, etc.}

Note that while Keynes is largely pro-market, he is anti-capitalist.
What he objects to is the central role that holders of capital play in
the then operative system. In his ideal world, capital is so plentiful
that it isn't a constraint. It's worth spending a bit of time I think on
what an anti-capitalist, pro-market system might look like.

Think for a bit about what's happened to the music business over the
last half century. Fifty years ago, you really needed capital to play in
the music business, and the people who had it served as gatekeepers.
Home recordings were dire, so you had to record in a studio, and these
were expensive. And no one had ability to press vinyl records at home,
so you have to pay one of the people who had a machine. And then you
needed some way to distribute those (simultaneously heavy and fragile)
across the country, probably to a retail store that was part of a chain.
The distribution is typically managed by a label, and getting more than
a handful of people to hear your music requires getting signed to one.
But you can't start your own if you expect to succeed; they are big
businesses.\sidenote{\footnotesize One of the biggest is even called Capitol.} There
are so many steps, and at every one of them you have to pay a fee.
Indeed, you have to find someone who is willing to take your money.

We can imagine a world that has none of those things. Technology has
meant home recordings are easier. And those recordings are distributable
in a way that does not require moving heavy or fragile things around. So
far so good.

Imagine a world where there was a site like Bandcamp, but it was run at
cost by a state or state-adjacent agency.\sidenote{\footnotesize I'm using
  `state-adjacent' here for things like USPS and UM that aren't part of
  the civil service, but aren't private actors either.} On this site
people could upload whatever music they wanted, and charge whatever they
wanted for it, and they got a large percentage of that money, with the
site doing no more than covering its costs of hosting and payments
processing.

That's almost, but not quite, Keynes's vision. His actual vision is that
the returns on capital are so low that someone will have been inspired
to set up Bandcamp-with-low-fees to get more than the pittance they can
earn in interest. Competition among people looking for places to invest
will drive the value of capital goods down, and the result will be that
there is a market for things made by actual producers (e.g., musicians),
and the infrastructure for it will be provided at really minimal cost.

This is the nirvana that we were supposed to get with low interest
rates. And two decades ago I might have argued that it was a plausible
story. But we had the low interest rate economy, and it didn't work that
way. So let's end with why it didn't.

\section{Euthanasia of the Rentier}\label{euthanasia-of-the-rentier}

Anglophone economics, indeed European economics, in a modern sense
starts with Adam Smith. And Smith, like most economists of his time, was
really opposed to rent. Businessmen have their flaws, to be sure, but
they are at least trying to provide goods and services. The landed
aristocracy are just taking a cut. Developing economic theory is part of
a broad campaign against these rent-seekers.\sidenote{\footnotesize Bankrupting the
  French monarchy and bringing on the French Revolution is also part of
  this campaign, though maybe not the most successful part.}

A rentier is a person living off their rents in general; not necessarily
rent of land or buildings, but possibly rent of capital goods. And
Keynes is saying that they are just as problematic as the early moderns
thought the landed aristocracy were. Keynes is on the side of
entrepreneurs against the people who want to charge them fees in order
to make businesses happen. It's just like Smith, except with the holders
of capital taking the place of the holders of land.

You can see the model Keynes has in mind. If interest rates are low, and
stay low, then the only thing that holders of capital will be able to do
to get any returns is to invest in real things. I'm pretty sure he has
factories in mind here as the kind of investment, and this will cause
complications. As the supply of these factories (or other investments)
goes up, the return on them goes down. If there are lots of studios to
rent, bands don't have to pay as much to rent studio time, and the same
is true for other industries. So eventually the return that the rentier
gets will fall away, because they can't get high interest rates, and
competition removes the returns they get from investment goods.

Unfortunately, none of this actually works in practice. Until very
recently I would have said it was all extremely plausible. But from the
global financial crisis through to the COVID crisis, we had extremely
low interest rates. (And even before 2008 interest rates were not
particularly high.) And none of what Keynes predicted came to pass. In
particular:

\begin{enumerate}
\def\labelenumi{\arabic{enumi}.}
\tightlist
\item
  There is still a shortage of capital. We're still begging/bribing
  people to build plants that make clean energy goods.
\item
  There is still a very good market for rentiers. The stock returns over
  that time have been uneven, but astronomical.
\end{enumerate}

So where did Keynes go wrong? And does it undermine the rest of his
theory? I don't have a good answer to the second part, though I have
some fears. I have three possible answers to the first part, which are
not entirely consistent. But they do add to the complications of the
models.

First, low interest rates lead people to go (to use a technical term)
batshit crazy. What happens when we have low interest rates forever is
not that people say ``Ah, this nice factory will return a steady 3\%,
let's build it.'' Instead they build Bored Ape Yacht Club. And I don't
have a rational choice or game theoretic story about this, but it
happened, and it's plausible that things like it are related to low
interest rates.

Second, low interest rates increase both the motivation and the
opportunity for financial crimes. If everyone is just getting 2\% on
their money, it's actually not that hard to run a literal Ponzi
scheme.\sidenote{\footnotesize Not legal advice, or financial advice, or anything
  else. Don't run Ponzi schemes!} Just promise 3\%, and make the 3\%
payments from the inflow of funds. There are details to be filled in,
but the basic story isn't that hard to make sense of. While Keynes's
model is in many ways more complicated than the standard models, it
doesn't have much place for doing crimes. But he's pushing to a world
where crimes are a big part of the economy.

Third, it's impossible to kill the dream of being a rentier. And low
interest rates encourage people to aim for future rentier status, even
if it's off in the future. So the standard model of how to get rich in
the 2010s was:

\begin{enumerate}
\def\labelenumi{\arabic{enumi}.}
\tightlist
\item
  Find an industry where there is a natural monopoly, probably due to
  consequences of attention being a limited resource.
\item
  Figure out a way that if you are the monopolist, you can earn rent by
  taking a cut of all the transactions in that industry.
\item
  Spend whatever it takes, most notably by providing goods and services
  at a loss, in order to become that monopolist.
\item
  Some time in the future, make large profits.
\end{enumerate}

Low interest rates mean that the profits in step 4 don't need to be as
large to justify the project. And, more importantly, they mean that the
profits in step 4 can be pushed off further into the future, and the
project still make sense.

So that's why in the 2010s we did not get people building up factories
for making wind turbines, solar panels, or EV batteries\sidenote{\footnotesize Or jet
  packs. We were promised jet packs.}, and instead got people throwing
money at Uber, MoviePass, WeWork, and all sorts of other strange ideas.

It's all a bit depressing really. You can see how Keynes's ideas would
work in a world where investment basically means building factories. But
in a world where investment means trying to become a monopolist (and
being willing to do some crimes before and after becoming one), low
interest rates aren't the panacea he hoped.

\section{For Next Time}\label{for-next-time}

We'll go back to the most obviously philosophical part of the book, the
discussion of uncertainty in chapter 12 and in the 1937 QJE article.



\end{document}
