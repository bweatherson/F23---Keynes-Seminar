% Options for packages loaded elsewhere
\PassOptionsToPackage{unicode}{hyperref}
\PassOptionsToPackage{hyphens}{url}
\PassOptionsToPackage{dvipsnames,svgnames,x11names}{xcolor}
%
\documentclass[
  11pt,
  letterpaper,
  DIV=11,
  numbers=noendperiod,
  oneside]{scrartcl}

\usepackage{amsmath,amssymb}
\usepackage{iftex}
\ifPDFTeX
  \usepackage[T1]{fontenc}
  \usepackage[utf8]{inputenc}
  \usepackage{textcomp} % provide euro and other symbols
\else % if luatex or xetex
  \ifXeTeX
    \usepackage{mathspec} % this also loads fontspec
  \else
    \usepackage{unicode-math} % this also loads fontspec
  \fi
  \defaultfontfeatures{Scale=MatchLowercase}
  \defaultfontfeatures[\rmfamily]{Ligatures=TeX,Scale=1}
\fi
\usepackage{lmodern}
\ifPDFTeX\else  
    % xetex/luatex font selection
  \setmainfont[Scale = MatchLowercase]{Scala Pro}
  \setsansfont[]{Scala Sans Pro}
  \ifXeTeX
    \setmathfont(Digits,Latin,Greek)[]{Scala Pro}
  \else
    \setmathfont[]{Scala Pro}
  \fi
\fi
% Use upquote if available, for straight quotes in verbatim environments
\IfFileExists{upquote.sty}{\usepackage{upquote}}{}
\IfFileExists{microtype.sty}{% use microtype if available
  \usepackage[]{microtype}
  \UseMicrotypeSet[protrusion]{basicmath} % disable protrusion for tt fonts
}{}
\makeatletter
\@ifundefined{KOMAClassName}{% if non-KOMA class
  \IfFileExists{parskip.sty}{%
    \usepackage{parskip}
  }{% else
    \setlength{\parindent}{0pt}
    \setlength{\parskip}{6pt plus 2pt minus 1pt}}
}{% if KOMA class
  \KOMAoptions{parskip=half}}
\makeatother
\usepackage{xcolor}
\usepackage[left=1in,marginparwidth=2.0666666666667in,textwidth=4.1333333333333in,marginparsep=0.3in]{geometry}
\setlength{\emergencystretch}{3em} % prevent overfull lines
\setcounter{secnumdepth}{-\maxdimen} % remove section numbering
% Make \paragraph and \subparagraph free-standing
\ifx\paragraph\undefined\else
  \let\oldparagraph\paragraph
  \renewcommand{\paragraph}[1]{\oldparagraph{#1}\mbox{}}
\fi
\ifx\subparagraph\undefined\else
  \let\oldsubparagraph\subparagraph
  \renewcommand{\subparagraph}[1]{\oldsubparagraph{#1}\mbox{}}
\fi


\providecommand{\tightlist}{%
  \setlength{\itemsep}{0pt}\setlength{\parskip}{0pt}}\usepackage{longtable,booktabs,array}
\usepackage{calc} % for calculating minipage widths
% Correct order of tables after \paragraph or \subparagraph
\usepackage{etoolbox}
\makeatletter
\patchcmd\longtable{\par}{\if@noskipsec\mbox{}\fi\par}{}{}
\makeatother
% Allow footnotes in longtable head/foot
\IfFileExists{footnotehyper.sty}{\usepackage{footnotehyper}}{\usepackage{footnote}}
\makesavenoteenv{longtable}
\usepackage{graphicx}
\makeatletter
\def\maxwidth{\ifdim\Gin@nat@width>\linewidth\linewidth\else\Gin@nat@width\fi}
\def\maxheight{\ifdim\Gin@nat@height>\textheight\textheight\else\Gin@nat@height\fi}
\makeatother
% Scale images if necessary, so that they will not overflow the page
% margins by default, and it is still possible to overwrite the defaults
% using explicit options in \includegraphics[width, height, ...]{}
\setkeys{Gin}{width=\maxwidth,height=\maxheight,keepaspectratio}
% Set default figure placement to htbp
\makeatletter
\def\fps@figure{htbp}
\makeatother

\setlength\heavyrulewidth{0ex}
\setlength\lightrulewidth{0ex}
\KOMAoption{captions}{tableheading}
\makeatletter
\@ifpackageloaded{caption}{}{\usepackage{caption}}
\AtBeginDocument{%
\ifdefined\contentsname
  \renewcommand*\contentsname{Table of contents}
\else
  \newcommand\contentsname{Table of contents}
\fi
\ifdefined\listfigurename
  \renewcommand*\listfigurename{List of Figures}
\else
  \newcommand\listfigurename{List of Figures}
\fi
\ifdefined\listtablename
  \renewcommand*\listtablename{List of Tables}
\else
  \newcommand\listtablename{List of Tables}
\fi
\ifdefined\figurename
  \renewcommand*\figurename{Figure}
\else
  \newcommand\figurename{Figure}
\fi
\ifdefined\tablename
  \renewcommand*\tablename{Table}
\else
  \newcommand\tablename{Table}
\fi
}
\@ifpackageloaded{float}{}{\usepackage{float}}
\floatstyle{ruled}
\@ifundefined{c@chapter}{\newfloat{codelisting}{h}{lop}}{\newfloat{codelisting}{h}{lop}[chapter]}
\floatname{codelisting}{Listing}
\newcommand*\listoflistings{\listof{codelisting}{List of Listings}}
\makeatother
\makeatletter
\makeatother
\makeatletter
\@ifpackageloaded{caption}{}{\usepackage{caption}}
\@ifpackageloaded{subcaption}{}{\usepackage{subcaption}}
\makeatother
\makeatletter
\@ifpackageloaded{sidenotes}{}{\usepackage{sidenotes}}
\@ifpackageloaded{marginnote}{}{\usepackage{marginnote}}
\makeatother
\ifLuaTeX
  \usepackage{selnolig}  % disable illegal ligatures
\fi
\IfFileExists{bookmark.sty}{\usepackage{bookmark}}{\usepackage{hyperref}}
\IfFileExists{xurl.sty}{\usepackage{xurl}}{} % add URL line breaks if available
\urlstyle{same} % disable monospaced font for URLs
\hypersetup{
  pdftitle={Week 5: Weight},
  pdfauthor={Brian Weatherson},
  colorlinks=true,
  linkcolor={black},
  filecolor={Maroon},
  citecolor={Blue},
  urlcolor={Blue},
  pdfcreator={LaTeX via pandoc}}

\title{Week 5: Weight}
\author{Brian Weatherson}
\date{2023-10-02}

\begin{document}
\maketitle
I thought I'd scribble down a few ideas on weight, and it got a bit
longer than I expected, and I ran out of time to sort the different
ideas out. So this isn't as organised as I'd like.

But that said, I think there's half a theory that emerges by the end of
the notes. It's that we should understand low weight as a high
propensity for valuations to change. Call an \(a\)-bet a bet that pays
£1 if \(a\), and 0 otherwise. If \(a/h\) is high weight, then we don't
expect our valuation of an \(a\)-bet to change much, unless the bet is
resolved.\sidenote{\footnotesize This caveat is annoying, but I think needed.} If
\(a/h\) is low weight, we expect our valuation of \(a\)-bets could
change rapidly.

This is meant as a generalisation of my old view that low weight meant a
high range of values of \(Pr(a | h)\) for \(Pr\) in the representor. If
that condition obtains, and we have a Caprice-like theory of decision
for imprecise credence\sidenote{\footnotesize Which I think Keynes endorses, and I
  certainly endorse.}, then it follows that the value of an \(a\)-bet
could easily change. After all, we might just capriciously change our
valuation. But it can also obtain even if \(a/h\) is numerical.

Note this isn't in any sense an \emph{interpretation} of Chapter 6, or
of the later writings about weight/uncertainty. What Keynes says in
Chapter 6 is incoherent, for reasons that I think Carnap was the first
to point out. What I've been interested in in the past, and am a bit
interested in here, is whether there is some coherent notion that does
what Keynes wants weight/uncertainty to do. And this idea about value
variation is my current best guess.

That said, I'm not at all sure this is right, or even plausible. And the
notes below are not a developed argument for that conclusion; I changed
my mind a couple of times while writing them and haven't altogether
edited the changes out. But hopefully there is a useful bit of
structuring there.

\section{A Story}\label{a-story}

Here's the basic idea Keynes runs in Chapter 6. Someone is doing an
investigation into a particular proposition \(p\). Let's say \(p\) is a
particular theory about the origins of COVID-19.\sidenote{\footnotesize There are wild
  conspiracies about this which I think should be set aside. But to the
  best of my knowledge, there are reasonable grounds for uncertainty
  about even the basic timeline. So it's a good example of something we
  can be very uncertain about.} The evidence here is a complete mess,
partially because the questions are very hard, and partially because the
Chinese government isn't exactly helpful.\sidenote{\footnotesize To be sure, I'm not
  sure which government would be.} So the inquirer might often get
evidence that supports \(p\) one day, and then opposes it the next.

Keynes has this mental picture of the investigator having an old
fashioned scale. When the evidence comes in, they classify it as either
for or against \(p\), and place it on the scale. How much the scale tips
says something about the probability of \(p\).

As the investigation continues, the probability might move in all sorts
of directions. One side or the other of the balance might have more
`weight' on it. But, says Keynes, the total evidence keeps going up. The
difference between the sides might change, and change direction. But the
sum of the sides keeps rising as we add more things to one or other pan.

That's the picture at least. And the thought is given that picture of
how evidence is accumulated, there is some interest in this value of the
sum of the evidence, not just its difference. Actually, maybe we'll come
back to this, is it difference or ratio that a balance beam is sensitive
to? I guess it's difference. Anyway, that's the picture.

\section{A Problem}\label{a-problem}

\subsection{Setting Up the Problem}\label{setting-up-the-problem}

Our investigator is bored one day, so they start flipping a coin. And,
being an investigator, they write the results down. First Tails, then
Heads, then three Tails in a row, and so on. The effect this has on the
investigation is zero; completely and utterly zero. This doesn't add
anything to the investigation. It is evidence; it could affect our
probability that the coin is biased. But it isn't evidence that's
relevant this investigation.\sidenote{\footnotesize If weight is just a function of
  how much total evidence one has, then we'd have \(V(a/h) = V(b/h)\)
  for any \(a, b\). And Keynes explicitly doesn't want that, and it
  becomes really important in the economic work that it isn't right.} So
it shouldn't add to the weight.

That sounds like an easy enough problem to solve. Say that
\(V(a/hh_1) = V(a/h)\) if \(a/hh_1 = a/h\). That is, if \(h_1\) doesn't
change the probability of \(a\), it doesn't change the weight of
\(a/h\). In this case, \(a\) is the old hypothesis about COVID-19, \(h\)
is the old evidence, and \(h_1\) is that this coin flip landed heads.

Unfortunately, there are several problems with this. Start with the
following problem. Imagine that there are two investigators looking into
\(p\). One of them is off sick for a couple of days. While they are out,
the first investigator discovers \(h_2\), which raises the probability
of \(p\). On Tuesday, they discover \(h_3\), which lowers the
probability of \(p\). On Wednesday, when the second investigator comes
back to work, and they are told of what was found while they were away,
i.e., \(h_2h_3\), the probability of \(p\) is unchanged. But the weight
has surely risen. After all, it rose on Monday and again on Tuesday. So
adding \(h_2h_3\) raises the weight. So sometimes, adding an independent
bit of evidence raises the weight.

There is a formal version of this problem. Assume that for any
\(x, y, z\), if \(y, z\) are logically equivalent, then
\(V(x/y) = V(x/z)\). That is, assume substitutivity of logical
equivalents within a weight operator. Then note that \(h_1\) is
logically equivalent to \((h_1 \vee p) \wedge (h_1 \vee \neg p)\). Then
we can reason as follows. I'll also assume something that I think we
can't do without, namely that on the right hand side of \(a/h\), set
theoretic union is conjunction. That is, adding two bits of evidence is
equivalent to adding their conjunction.\sidenote{\footnotesize Keynes certainly writes
  as if this is a trivial assumption in many places, and I don't think
  it's avoidable.}

\begin{align*}
V(a/h) &= V(a/hh_1) \\
   &= V(a/h \wedge ((h_1 \vee p) \wedge (h_1 \vee \neg p)) \\
   &= V(a/h \cup \{h_1 \vee p\} \cup \{h_1 \vee \neg p\}) \\
   &> V(a/h \cup \{h_1 \vee p\}) \\
   &> V(a/h)
\end{align*}

And oops, we have a contradiction. How could we avoid this? I can see
two options that are relatively simple, and which don't seem attractive,
and one that is less simple, but might do the trick.

\subsection{Keynes's Solution}\label{keyness-solution}

Keynes says, without quite realising it, that the argument fails at the
first step. He says that independent evidence can add to weight as long
as it's (equivalent to) a conjunction of two things that change the
probability. And the problem, as Carnap pointed out and as Cohen also
noted, was that everything is equivalent to the conjunction of two
things that change the probability. So on Keynes's solution, flipping
the coin does add weight.

\subsection{Denying Equivalence}\label{denying-equivalence}

Is there a way out of this? I'm not sure, but let's try some things. We
could play with the idea that weight is a function of arguments, and
arguments are made of sentences, not propositions.\sidenote{\footnotesize I think
  Lloyd Humberstone defined an argument place as \emph{congruent} if it
  allowed for substitution of logical equivalents, but I'm not 100\%
  sure I've got the term right, and I'd have to look up the reference.}
We don't have to say that if \(y, z\) are equivalent, then
\(V(x/y) = V(x/z)\). We definitely do have that \(x/y = x/z\), but
Keynes explicitly rejects the inference from that to
\(V(x/y) = V(x/z)\).

Then we might say that evidence adds no weight if it is independent,
and, if it is of the form \(A \wedge B\), both \(A\) and \(B\) are
independent.

I don't know how much trouble this will raise elsewhere. I kind of think
this might work, but it's a big change at least to how we think about
probability in contemporary work.

\subsection{Weight as Projection}\label{weight-as-projection}

Here's the thing about the coin flip. It's not just that it doesn't make
a difference to the probability of \(p\) now. We know (in some intuitive
sense of `know') that it won't make a difference going forward. It's not
just irrelevant given \(h\), it's irrelevant given any reasonable
extension of \(h\).

Compare this kind of case. We're gossiping and trying to figure out if
Jack and Jill are dating. Someone adds that they saw Jack at a couple of
places last night; having hamburgers early in the night at Blimpy and
then at the comedy club later. That doesn't seem like it moves the
probability that Jack and Jill are dating one way or the other. Then
someone else adds that they saw Jill at both these places. That could be
a coincidence, but that's a bit of interesting information.

So what's happened there is the first bit of information, where Jack
was, is arguably independent of whether Jack and Jill are dating. And
similarly on its own, the information about where Jill was, is also
independent. But put those two pieces of information together, and now
we have something. Hardly conclusive proof, but enough of a probability
shifter to make fun gossip.

Intuitively, it seems that we want to say is that new evidence raises
the weight if it changes the probability, or it could change the
probability in the future. This doesn't on it's own solve the problem.
Go back to the coin flip, and let \(h_1\) be a proposition about the
coin flip. For any \(a\), we could learn \(h_1 \supset a\), and then
\(h_1\) would be relevant. I think what's going on is that while we
logically could learn that, it's not going to happen. There is no
reasonable world in which we learn a material conditional connecting a
coin flip to the origins of COVID, and nothing else. So here's my worked
out hypothesis.

\begin{itemize}
\tightlist
\item
  Given \(h\), there is a set \(S\) of things that are possible items of
  future evidence.
\item
  \(V(a/hh_1) > V(a/h)\) iff for some \(e \in S: a/heh_1 \neq a/he\).
\end{itemize}

This won't quite work for a reason we'll get to below, but it's
progress. I think it helps with the two investigators problem. We could
learn on Thursday that there was something wrong with the way the Monday
or Tuesday investigations were conducted. And then \(h_2h_3\) would be
relevant to \(p\). So I think this does the job that Keynes introduced
general irrelevance to do. The problem is that there are three other
issues that I don't think Keynes has the start of an answer to, and I'm
not really sure what to say about them.

\section{Three More Problems}\label{three-more-problems}

\subsection{How Many Scales?}\label{how-many-scales}

The issue of how the weighing metaphor works is one of the biggest
topics in philosophy in the last 15 years. The starting point for this
is Joshua Gert's paper ``Normative Strength and the Balance of
Reasons''.\sidenote{\footnotesize \emph{Philosophical Review}, 2007} In 2016 there was
a useful edited collection on the topic by Errol Lord and Barry
Maguire.\sidenote{\footnotesize Called simply \emph{Weighing Reasons}, and Jussi
  Suikkanen has a
  \href{https://ndpr.nd.edu/reviews/weighing-reasons-2/}{useful review
  of it in NDPR}.} And there has been a flood of work since. There are
two issues that come up in this literature that I'll briefly mention.

One important bit of background is that in the usual division of
philosophical topics, this has all been taking place in what we think of
as ethics, not what we think of as epistemology, let alone probability.
But I think it's fundamentally the same set of problems. The folks in
the ethics literature have all been talking about weighing
\emph{reasons}, but it's commonly (and I think correctly) thought that
evidence is a kind of reason, so weighing evidence is the same set of
issues.

The topic that has been central to this debate in ethics I don't think
is relevant here. Assume that you think that there are actions that are
permissible but not obligatory. And not just because they are tied for
best with some other actions; they are permissible but other actions are
\emph{better}. What should we say about the weight of reasons in favour
of the less good, but permissible, action? If it's less than the weight
of reasons in favour of the other action, then why is it permissible? If
it's the same as the weight of reasons in favour of the other action,
why is the other better?

The standard solution here is to say that there are (in the metaphor)
two kinds of scales: a permission scale and an obligation scale. And the
options are tied on one and not on the other. I don't really love this
move; it seems like the two should be integrated more. But anyway, I
don't think it's relevant here since I don't think this division between
permission and obligation is relevant.\sidenote{\footnotesize Maybe that's going to be
  wrong at the end, given what we say about the \emph{General Theory}.
  But I think it's a separate issue.}

\subsection{Adding Weights}\label{adding-weights}

Keynes seems to miss the relevance of a deeply Moorean point here:
evidence satisfies an organic unity principle. (Confirmation is
holistic, in slightly more updated lingo, but I prefer the Moorean
terms.) Think again about the Jack and Jill example. What's the weight
of each piece of evidence? Honestly, not that much. But the weight of
their sum is, intuitively, substantial.

This is one of the big points in the recent ethics literature, and I
think it does create a problem here. The metaphor relies on the idea
that bits of evidence are like blocks you put on a scale. But they don't
satisfy an organic unity principle. I think this is a reason to be
sceptical of the whole `weighing' metaphor. And without the metaphor,
the idea of summing the weights of the two sides is a bit dubious.

\subsection{Defeaters}\label{defeaters}

This is in some sense a version of the previous problem, but it's
distinctive enough to be worth it's own note. Sometimes, evidence is
\textbf{defeated}. Here's a famous real-life example.\sidenote{\footnotesize At least,
  it actually happened in the HBO series. I think it might have also
  happened in actual incident.}

After Chernobyl melted down, they were understandably worried about the
radiation levels in the reactor and the surrounding area. So they got
out the trusty Measure-Background-Radiation-Meter, and it read 3.6.
That's not great; it should be 0. But it's not bad; it will take months
of exposure at 3.6 to develop radiation linked illnesses. Let \(a\) be
that people are in immediate danger, \(h\) the background information
(i.e., that there was a meltdown), and \(h_1\) the reading of 3.6. Then
it seems \(a/hh_1 < a/h\), and that \(V(a/hh_1) > V(a/h)\).

If you've seen the show, or read about Chernobyl, you know what comes
next. It turns out the machine was capped out at 3.6; it simply wouldn't
show any higher.\sidenote{\footnotesize I read somewhere that the cap was at 1/1000
  per second, and the 3.6 was the rate per hour at 1/1000 per second.
  Capping at 3.6 seemed odd to me, but this might explain it.} Let
\(h_2\) be that the machine does not have a reading above 3.6. Then
\(a/hh_1h_2 \approx a/h\). But, and this is the important part, arguably
\(V(a/hh_1h_2) \approx V(a/h)\). It's not just that the new evidence
\(h_2\) pushed in the other direction to \(h_1\). It's that it showed
\(h_1\) wasn't really evidence in the first place.

Keynes doesn't have any way of capturing that. His core idea is that new
evidence, i.e., evidence that changes our view, adds to the weight. And
I don't think that's the case. Sometimes it tells us that what we
thought was evidence was not in fact evidence.

There is a small literature on how to model this kind of \emph{defeat}
in probabilistic terms.\sidenote{\footnotesize Matt Kotzen, who's visiting UM this
  week, has \href{https://matthewkotzen.net/s/AFAOEDrevised.pdf}{a
  paper} going over some of the details.} I don't know that literature
particular well, but I didn't get the sense that there was a clear way
of getting the models to work. We need a way of differentiating evidence
that undermines previous evidence, from evidence that simply points in
the opposite direction. Imagine if instead of \(h_2\) we'd discovered
\(h_3\), that people near the reactor had a particular condition that
meant even at 3.6 they were in immediate danger. That would raise the
probability of \(a\) back up, but intuitively in a way that did increase
the weight. Just looking at probability kinematics, adding \(h_2\) and
adding \(h_3\) have the same effect, but one decreases weight, the other
increases it.

Maybe the idea I floated earlier about a future-directed view could
help? In the real world, where we learned \(h_2\), we think \(h_1\) is
useless, \emph{and will remain useless in any realistic scenario}. Maybe
after learning \(h_1h_3\), there is still some importance to \(h_1\) in
this inquiry? At least, perhaps it could be relevant given some future
learning?

\section{Weight and Imprecise
Probabilities}\label{weight-and-imprecise-probabilities}

In a long ago paper\sidenote{\footnotesize ``Keynes, Uncertainty and Interest Rates'',
  in \emph{Cambridge Journal of Economics}} I argued that we should
understand \(V(a/h)\) using imprecise probabilities. If \(S\) is the set
of prior probability functions, then
\(V(a/h) = max(Pr(a|h)) - min(Pr(a|h))\), where the \(max\) and \(min\)
range over the members of \(S\). This is massively different to how
Keynes understood weight, but at the time I thought it was the best way
to capture what was right about the view.

One problem is that because of dilation, on my view weight can go down.
At the time I thought that was fine because defeaters imply that weight
can go down. But now I don't think this really works. The cases like
Chernobyl where weight goes down, are very different to the Gibbon
example of dilation.

Another problem is that it implies something Keynes explicitly rejects,
namely that \(a/h_1 = b/h_2\) implies that \(V(a/h_1) = V(b/h_2)\). And
I now think the examples in paragraph 4 of chapter 6 are a better case
for rejecting that than I'd previous realised.

Here's the case I have in mind. Imagine we have some \(\phi\) which
(given background \(h\)) satisfies all the conditions we could want for
Indifference, and (also given \(h\)) the possible satisfiers of \(\phi\)
are \(x,y,z\). Then \(\phi(x)/h = 1/3\). Now imagine we also have a die,
and we've carefully measured it to see that it's fair, and tested some
automated rolling mechanism many times and seen neither bias nor pattern
in the rolls. Assume this is also all part of \(h\). Let \(b\) be that
the die will land 1 or 2 on its next roll. Then \(b/h = 1/3\) also. But
\(V(b/h) > V(\phi(x)/h)\), since the former is based on a huge body of
evidence, and the latter is based on Indifference.

\section{Weight in the Economics
Work}\label{weight-in-the-economics-work}

Assume we can somehow fix these problems and get a notion of weight that
works. Is it theoretically useful? In the \emph{Treatise}, Keynes is
unsure here. He says that it seems interesting, but he can't see how to
give it an application. In the \emph{General Theory}, he does give it an
application, and an incredibly interesting one. We are going to talk
about this much more in later weeks, but I wanted to flag it here
because it's really a pivot point of this course. Here's a key paragraph
from the remarkable
\href{http://brian.weatherson.org/general-theory/12.html}{chapter 12 of
the \emph{General Theory}}.\sidenote{\footnotesize The footnotes that follow are my
  annotations; they are not in the original. I've tried to copy in all
  the italics from the originals, but I'm not sure I got all of them. He
  does use a lot of italics.}

\begin{quote}
It would be foolish, in forming our expectations, to attach great weight
to matters which are very uncertain. It is reasonable, therefore, to be
guided to a considerable degree by the facts about which we feel
somewhat confident, even though they may be less decisively relevant to
the issue than other facts about which our knowledge is vague and
scanty. For this reason the facts of the existing situation enter, in a
sense disproportionately, into the formation of our long-term
expectations; our usual practice being to take the existing situation
and to project it into the future, modified only to the extent that we
have more or less definite reasons for expecting a change.\sidenote{\footnotesize The
  model is that if we have low weight of evidence for some factor, we
  act as if it is 0. This isn't perfectly rational, but it's pretty
  good.}
\end{quote}

And at the end of the first sentence, we get this footnote.

\begin{quote}
By ``very uncertain'' I do not mean the same thing as ``improbable''.
Cf. my \emph{Treatise on Probability}, chap.~6, on ``The Weight of
Arguments''.
\end{quote}

Why is chapter 12 remarkable? Because it doesn't read much like an
economics paper, at least not a mainstream one. There are no equations.
There are hardly any numbers. It's all about how no one knows anything,
and we're all guessing, and we have to basically pretend that we're in a
position to do anything like probabilistic reasoning.

The idea that we know very little about what we need in order to invest
is a frequently recurring theme in the chapter. I could quote
practically the whole thing, but here are a couple more relevant
passages.

\begin{quote}
The outstanding fact is the extreme precariousness of the basis of
knowledge on which our estimates of prospective yield have to be made.
Our knowledge of the factors which will govern the yield of an
investment some years hence is usually very slight and often negligible.
If we speak frankly, we have to admit that our basis of knowledge for
estimating the yield ten years hence of a railway, a copper mine, a
textile factory, the goodwill of a patent medicine, an Atlantic liner, a
building in the City of London amounts to little and sometimes to
nothing; or even five years hence. In fact, those who seriously attempt
to make any such estimate are often so much in the minority that their
behaviour does not govern the market.\sidenote{\footnotesize Example of low weight:
  investment yield 5-10 years hence on mines, factories, ships,
  medicines, \emph{commercial} real estate.}
\end{quote}

And a bit later (this crosses sections 3 and 4 of the chapter), after a
discussion of how stock market prices bounce so much during the day:

\begin{quote}
How then are these highly significant daily, even hourly, revaluations
of existing investments carried out in practice?\sidenote{\footnotesize It seems
  quaint to this of stock markets re-evaluating on an hourly basis.}

In practice we have tacitly agreed, as a rule, to fall back on what is,
in truth, a convention. The essence of this convention --- though it
does not, of course, work out quite so simply --- lies in assuming that
the existing state of affairs will continue indefinitely, except in so
far as we have specific reasons to expect a change.\sidenote{\footnotesize It's going
  to turn out that, quite surprisingly, this is in fact a Lewisian
  convention.} This does not mean that we really believe that the
existing state of affairs will continue indefinitely. We know from
extensive experience that this is most unlikely. The actual results of
an investment over a long term of years very seldom agree with the
initial expectation. Nor can we rationalise our behaviour by arguing
that to a man in a state of ignorance errors in either direction are
equally probable, so that there remains a mean actuarial expectation
based on equi-probabilities. For it can easily be shown that the
assumption of arithmetically equal probabilities based on a state of
ignorance leads to absurdities.\sidenote{\footnotesize Note the callback to chapter 4
  of the \emph{Treatise}.} We are assuming, in effect, that the existing
market valuation, however arrived at, is uniquely \emph{correct} in
relation to our existing knowledge of the facts which will influence the
yield of the investment, and that it will only change in proportion to
changes in this knowledge; though, philosophically speaking it cannot be
uniquely correct, since our existing knowledge does not provide a
sufficient basis for a calculated mathematical expectation.\sidenote{\footnotesize Here
  he connects weight and imprecise probability; this is potentially
  important.} In point of fact, all sorts of considerations enter into
the market valuation which are in no way relevant to the prospective
yield.

Nevertheless the above conventional method of calculation will be
compatible with a considerable measure of continuity and stability in
our affairs, \emph{so long as we can rely on the maintenance of the
convention}.

For if there exist organised investment markets and if we can rely on
the maintenance of the convention, an investor can legitimately
encourage himself with the idea that the only risk he runs is that of a
genuine change in the news \emph{over the near future}, as to the
likelihood of which he can attempt to form his own judgment, and which
is unlikely to be very large.\sidenote{\footnotesize What makes this a Lewisian
  convention is that we're all better off if we all follow it because we
  all get extra \emph{liquidity}.} For, assuming that the convention
holds good, it is only these changes which can affect the value of his
investment, and he need not lose his sleep merely because he has not any
notion what his investment will be worth ten years hence.
\end{quote}

In chapter 17 we get a return to the same theme. Indeed, there is a
footnote in the middle of what I'm about to quote that refers back to
the footnote which refers back to the \emph{Treatise}. The theme of the
chapter is that capital goods (including factories, houses, stocks of
commodities, and money) have three important characteristics: their
yield, their carrying costs, and their `liquidity-premium'. The latter
is the amount people will pay in order to have something they can
quickly convert into something else. Here's the quote, this time
somewhat elided.

\begin{quote}
Consider, for example, an economy in which there is no asset for which
the liquidity-premium is always in excess of the carrying-costs; which
is the best definition I can give of a so-called ``non-monetary''
economy. There exists nothing, that is to say, but particular
consumables and particular capital equipments \ldots{} In such an
economy capital equipments will differ from one another \ldots{} (c) in
the rapidity with which the wealth embodied in them can become
``liquid'', in the sense of producing output, the proceeds of which can
be re-embodied if desired in quite a different form. The owners of
wealth will then weigh the lack of ``liquidity'' of different capital
equipments in the above sense as a medium in which to hold wealth
against the best available actuarial estimate of their prospective
yields after allowing for risk. The liquidity-premium, it will be
observed, is partly similar to the risk-premium, but partly different;
--- the difference corresponding to the difference between the best
estimates we can make of probabilities and the confidence with which we
make them.\sidenote{\footnotesize In the original Keynes has a footnote here linking
  back to the discussion of weight.} \ldots{} There is, clearly, no
absolute standard of ``liquidity'' but merely a scale of liquidity
\ldots{} The conception of what contributes to ``liquidity'' is a partly
vague one, changing from time to time and depending on social practices
and institutions.\sidenote{\footnotesize In our world, stocks in large listed
  companies are incredibly liquid for 6.5 hours 5 days a week, and
  somewhat less liquid other times. I've heard of people who use the
  changing liquidity of stocks as the Friday close approaches as part of
  successful trading strategies.} The order of preference in the minds
of owners of wealth in which at any given time they express their
feelings about liquidity is, however, definite and is all we require for
our analysis of the behaviour of the economic system. \ldots{} It may be
that in certain historic environments the possession of land has been
characterised by a high liquidity-premium in the minds of owners of
wealth\ldots{}
\end{quote}

And in a contribution the following year to a symposium on the book, he
writes:

\begin{quote}
By ``uncertain'' knowledge, let me explain, I do not mean merely to
distinguish what is known for certain from what is only
probable.\sidenote{\footnotesize The talk of weight has dropped out, but `uncertain'
  seems a callback to this.} The game of roulette is not subject, in
this sense, to uncertainty; nor is the prospect of a Victory bond being
drawn.\sidenote{\footnotesize Two examples of zero uncertainty, both with defined
  numerical probabilities.} Or, again, the expectation of life is only
slightly uncertain.\sidenote{\footnotesize Compare the discussion of Gibbon in chapter
  4 of the \emph{Treatise}.} Even the weather is only moderately
uncertain.\sidenote{\footnotesize Uncertainty, in the 1937 sense, comes in degrees,
  like weight.} The sense in which I am using the term is that in which
the prospect of a European war is uncertain,\sidenote{\footnotesize Example of
  uncertainty: prospect of European war} or the price of copper and the
rate of interest twenty years hence\sidenote{\footnotesize Example of uncertainty:
  economic fundamentals 20 years hence}, or the obsolescence of a new
invention\sidenote{\footnotesize Example of uncertainty: obsolescence of new invention},
or the position of private wealth-owners in the social system in
1970\sidenote{\footnotesize Example of uncertainty: whether there will be a Bolshevik
  revolution}. About these matters there is no scientific basis on which
to form any calculable probability whatever.\sidenote{\footnotesize Again, there's a
  connection here to non-numerical probability} \textbf{We simply do not
know} (emphasis added). Nevertheless, the necessity for action and for
decision compels us as practical men to do our best to overlook this
awkward fact and to behave exactly as we should if we had behind us a
good Benthamite calculation of a series of prospective advantages and
disadvantages, each multiplied by its appropriate probability, waiting
to be summed.

How do we manage in such circumstances to behave in a manner which saves
our faces as rational, economic men? We have devised for the purpose a
variety of techniques, of which much the most important are the three
following:

(1) We assume that the present is a much more serviceable guide to the
future than a candid examination of past experience would show it to
have been hitherto. In other words we largely ignore the prospect of
future changes about the actual character of which we know
nothing.\sidenote{\footnotesize This is a callback to the start of chapter 12.}

(2) We assume that the existing state of opinion as expressed in prices
and the character of existing output is based on a \emph{correct}
summing up of future prospects, so that we can accept it as such unless
and until something new and relevant comes into the picture.

\begin{enumerate}
\def\labelenumi{(\arabic{enumi})}
\setcounter{enumi}{2}
\tightlist
\item
  Knowing that our own individual judgment is worthless, we endeavour to
  fall back on the judgment of the rest of the world which is perhaps
  better informed. That is, we endeavour to conform with the behaviour
  of the majority or the average. The psychology of a society of
  individuals each of whom is endeavouring to copy the others leads to
  what we may strictly term a \emph{conventional} judgment.
\end{enumerate}

Now a practical theory of the future based on these three principles has
certain marked characteristics. In particular, being based on so flimsy
a foundation, it is subject to sudden and violent changes. The practice
of calmness and immobility, of certainty and security, suddenly breaks
down. New fears and hopes will, without warning, take charge of human
conduct. The forces of disillusion may suddenly impose a new
conventional basis of valuation. All these pretty, polite techniques,
made for a well-panelled Board Room and a nicely regulated market, are
liable to collapse. At all times the vague panic fears and equally vague
and unreasoned hopes are not really lulled, and lie but a little way
below the surface.

Perhaps the reader feels that this general, philosophical disquisition
on the behaviour of mankind is somewhat remote from the economic theory
under discussion. But I think not.
\end{quote}

I want to note a few points about this, which we can discuss more in
seminar.

\begin{enumerate}
\def\labelenumi{\arabic{enumi}.}
\tightlist
\item
  There is a connection of low weight, in the sense of the
  \emph{Treatise} to the very uncertain, in the sense of the
  \emph{General Theory}. I don't see any other way to read that
  footnote.
\item
  We get a lot of examples of ``uncertain'' things. And, to my eye, they
  all fit the high value for max minus min probability in the set. Even
  the `moderate' uncertainty of the weather fits the model; max minus
  min is not zero, but it's not great.
\item
  Moreover, we see both in the \emph{General Theory} and the follow up
  paper that Keynes connected uncertainty with the absence of numerical
  probabilities. There are two bits of textual evidence for that, and
  the examples all fit the pattern. Note here that he drops the one
  example that pulls them apart in the \emph{Treatise}: a numerical
  probability based on a direct application of the Principle of
  Indifference.
\item
  But that said, the actual practical role of uncertainty concerns its
  illiquidity. And illiquidity does not, as far as I can tell, rely on
  non-numerical probabilities.
\item
  Moreover, a dedicated believer in numerical probabilities could say
  that yes, sometimes we have more evidence for one thing rather than
  another, and that, as a result of that, the former thing will have
  less violent fluctuations in its probability than the other. This is a
  version of the Jeffrey response to Popper.
\end{enumerate}

So I dimly suspect that Keynes has run together two related, but
philosophically distinct views here. One is that for some investments,
we do not have ``behind us a good Benthamite calculation of a series of
prospective advantages and disadvantages, each multiplied by its
appropriate probability, waiting to be summed.'' The other is that even
between investments that do have that characteristic, some are more or
less liquid than others. And both things affect the desire for
investment.

All that said, and to be a bit fairer to Keynes than the last paragraph
potentially was, once we've distinguished the two notions, we can see
why they'd be connected in the case that's of most interest to him.
Start with a case where we only are concerned with non-numerical
probabilities, and not at all with liquidity and/or potential changes in
valuation as new evidence comes in. We'll find situations like the
following. (This is from later in Chapter 12 of the \emph{General
Theory}.)

\begin{quote}
Even apart from the instability due to speculation, there is the
instability due to the characteristic of human nature that a large
proportion of our positive activities depend on spontaneous optimism
rather than on a mathematical expectation, whether moral or hedonistic
or economic. Most, probably, of our decisions to do something positive,
the full consequences of which will be drawn out over many days to come,
can only be taken as a result of animal spirits\sidenote{\footnotesize This is the
  only occurrence of this phrase in the \emph{General Theory}, and it
  became one of the most quoted parts of the book.} --- of a spontaneous
urge to action rather than inaction, and not as the outcome of a
weighted average of quantitative benefits multiplied by quantitative
probabilities.
\end{quote}

In modern economies, we don't typically invest in copper mines. We
invest in copper mining companies, as traded on the stock market. And
we're all guessing all the time; we're not doing anything like EV
maximisation. But that's fine, we'll just ride our luck. The problem is
that if everyone else is doing the same thing, then the value of our
stock in copper-mining companies might collapse not because of any
evidence, but because of a change in animal spirits. And that's a new
risk we take. And it's a risk that affects not just the potential return
of our investment at the end of the investment period, but the
possibility of using the invested money to pay for car repairs, or a
hospital bill, during the investment period. That is, it affects the
liquidity of the investment. And that's because there is a distinctive
kind of evidence, evidence that animal spirits are changing, to which
the investment is sensitive. And this is a kind of evidence that we are
very likely to get, because animal spirits are not, in their nature,
very stable. So even if (as I half suspect) Keynes is running two
different notions together, in practice they are likely to coincide in
cases he's most interested in.

And if we aren't interpreting Keynes, but finding some modern version of
his ideas, there is a natural blend of these two ideas that he ran
together. I said that for Keynes there are two characteristics of low
weight, which perhaps he did not cleanly separate even in his mind:

\begin{enumerate}
\def\labelenumi{\arabic{enumi}.}
\tightlist
\item
  High range of probabilities in the representor;
\item
  High fluctuations in probability over reasonably likely events.
\end{enumerate}

There is one way to merge those. Those two things are individually
sufficient, and jointly necessary\sidenote{\footnotesize I think I've written
  individually necessary and jointly sufficient a billion times in
  various things; that's the first occasion I can remember for this
  direction.} for the following:

\begin{itemize}
\tightlist
\item
  High fluctuations in value of bets on the proposition in reasonably
  likely events.
\end{itemize}

So maybe if we take low weight to be high fluctuations in value, we get
something that certainly isn't Keynes's view - Keynes's view is
incoherent and this is coherent - but something that captures what he
was aiming for in his view, and does the economic work he needs it to
do.



\end{document}
